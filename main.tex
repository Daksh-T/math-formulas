\documentclass{article}
\usepackage{amsmath}
\usepackage{amssymb}

\title{Mathematics Formula List}
\author{}
\date{}

\begin{document}

\maketitle

\section{General Mathematical Rules}

\subsection{Arithmetic Mean and Geometric Mean}

Arithmetic Mean is always greater than or equal to Geometric Mean. For example, for three numbers \(a_1\), \(a_2\), and \(a_3\):

\begin{equation}
\text{Arithmetic Mean} = \frac{a_1 + a_2 + a_3}{3}
\end{equation}

\begin{equation}
\text{Geometric Mean} = \sqrt[3]{a_1 \cdot a_2 \cdot a_3}
\end{equation}

\section{Sets}

\begin{equation}
n(P \cup Q) = n(P) + n(Q) – n(P \cap Q)
\end{equation}

\begin{equation}
n(P \cup Q \cup R) = n(P) + n(Q) + n(R) - n(P \cap Q) - n(Q \cap R) - n(R \cap P) + n(P \cap Q \cap R)
\end{equation}

\begin{equation}
(A \cup B)^{'} = A^{'} \cap B^{'}
\end{equation}

\begin{equation}
(A \cap B)^{'} = A^{'} \cup B^{'}
\end{equation}

\textbf{Example of Cartesian Product:}

Let

\[
A = \{a, b\}
\]

\[
B = \{1, 2\}
\]

\[
C = \{x, y\}
\]

Then

\[
A \times B \times C = \{(a, 1, x), (a, 1, y), (a, 2, x), (a, 2, y), (b, 1, x), (b, 1, y), (b, 2, x), (b, 2, y)\}
\]

\section{Trigonometry}

\subsection{Sum and Difference Formulas}

\begin{align}
\sin(x+y) &= \sin(x)\cos(y) + \cos(x)\sin(y) \\
\cos(x+y) &= \cos(x)\cos(y) - \sin(x)\sin(y) \\
\tan(x+y) &= \frac{\tan(x)+\tan(y)}{1-\tan(x)\tan(y)} \\
\sin(x-y) &= \sin(x)\cos(y) - \cos(x)\sin(y) \\
\cos(x-y) &= \cos(x)\cos(y) + \sin(x)\sin(y) \\
\tan(x-y) &= \frac{\tan(x)-\tan(y)}{1+\tan(x)\tan(y)}
\end{align}

\subsection{Double Angle Formulas}

\begin{align}
\sin(2x) &= 2\sin(x)\cos(x) = \frac{2\tan(x)}{1+\tan^2(x)} \\
\cos(2x) &= \cos^2(x) - \sin^2(x) = \frac{1-\tan^2(x)}{1+\tan^2(x)} \\
\cos(2x) &= 2\cos^2(x) - 1 = 1 - 2\sin^2(x) \\
\tan(2x) &= \frac{2\tan(x)}{1-\tan^2(x)} \\
\sec(2x) &= \frac{\sec^2(x)}{2-\sec^2(x)} \\
\csc(2x) &= \frac{\sec(x)\csc(x)}{2}
\end{align}

\subsection{Triple Angle Formulas}

\begin{align}
\tan(3x) &= \frac{3\tan(x)-\tan^3(x)}{1-3\tan^2(x)} \\
\sin(3x) &= 3\sin(x)-4\sin^3(x) \\
\cos(3x) &= 4\cos^3(x)-3\cos(x)
\end{align}

\subsection{Product-to-Sum and Sum-to-Product Formulas}

\begin{align}
\sin(x)\cos(y) &= \frac{\sin(x+y) + \sin(x-y)}{2} \\
\cos(x)\cos(y) &= \frac{\cos(x+y) + \cos(x-y)}{2} \\
\sin(x)\sin(y) &= \frac{\cos(x-y) - \cos(x+y)}{2} \\
\sin(x)+\sin(y) &= 2\sin\left(\frac{x+y}{2}\right)\cos\left(\frac{x-y}{2}\right) \\
\sin(x)-\sin(y) &= 2\cos\left(\frac{x+y}{2}\right)\sin\left(\frac{x-y}{2}\right) \\
\cos(x)+\cos(y) &= 2\cos\left(\frac{x+y}{2}\right)\cos\left(\frac{x-y}{2}\right) \\
\cos(x)-\cos(y) &= -2\sin\left(\frac{x+y}{2}\right)\sin\left(\frac{x-y}{2}\right)
\end{align}

\subsection{Power Reduction Formulas}

\begin{equation}
\cos^4 \theta + \sin^4 \theta = 1 - 2\sin^2 \theta \cos^2 \theta
\end{equation}

\begin{equation}
\sin^6 \theta + \cos^6 \theta = 1 - 3\sin^2\theta \cos^2 \theta
\end{equation}

\subsection{Cotangent Formulas in Terms of Angle Sum}

If \(A + B + C = 180^{\circ}\), we have:

\begin{equation}
\cot (A + B) = \cot (\pi - C)
\end{equation}

\begin{equation}
\frac{\cot A \cot B - 1}{\cot A + \cot B} = -\cot C
\end{equation}

\subsection{Formulas Expressing Products of Trigonometric Functions in Sum/Difference of Angles}

\begin{align}
2\sin A\cos B &= \sin(A+B) + \sin(A-B) \\
2\cos A\sin B &= \sin(A+B) - \sin(A-B) \\
2\cos A\cos B &= \cos(A+B) + \cos(A-B) \\
2\sin A\sin B &= \cos(A-B) - \cos(A+B)
\end{align}

\subsection{Formulas Involving Multiple of an Angle}

\begin{align}
4\sin(60-\theta) \cdot \sin(\theta) \cdot \sin(60+\theta) &= \sin(3\theta) \\
4\cos(60-\theta) \cdot \cos(\theta) \cdot \cos(60+\theta) &= \cos(3\theta) \\
\tan(60-\theta) \cdot \tan(\theta) \cdot \tan(60+\theta) &= \tan(3\theta)
\end{align}

\subsubsection{Note}

Whenever the sum of squares of two numbers is 1, and one of them is assumed to be \(\sin(\theta)\), then the other is \(\cos(\theta)\). Also,

\[
-\sqrt{a^2+b^2} \leq a\sin x + b\cos x \leq \sqrt{a^2+b^2}
\]

\section{Logarithms}

\begin{equation}
y = a^x \iff x = \log_a(y) \quad \text{where } x > 0, a > 0, \text{ and } a \neq 1
\end{equation}

\begin{equation}
\log_a(1) = 0
\end{equation}

\begin{equation}
\log_a(a) = 1
\end{equation}

\begin{equation}
\log_a(x) + \log_a(y) = \log_a(xy)
\end{equation}

\begin{equation}
\log_a(x) - \log_a(y) = \log_a\left(\frac{x}{y}\right)
\end{equation}

\begin{equation}
n \log_a(x) = \log_a(x^n)
\end{equation}

\begin{equation}
\log_a^n(x) = \frac{1}{n} \log_a(x)
\end{equation}

\begin{equation}
(a^n)^y = x \Rightarrow y = \frac{1}{n} \log_a(x)
\end{equation}

\begin{equation}
\log_{10}(x) = \log(x)
\end{equation}

\begin{equation}
\log_{e}(x) = \ln(x) \quad \text{where } e \approx 2.71 \text{ (Euler's number, an irrational constant)}
\end{equation}

\begin{equation}
\log_a(b) = \frac{1}{\log_b(a)}
\end{equation}

\begin{equation}
\log_a(b) = \frac{\log_c(b)}{\log_c(a)} = \frac{\log(b)}{\log(a)} = \frac{\ln(b)}{\ln(a)}
\end{equation}

\textbf{Some common log values:}

\begin{align*}
\log(2) &= 0.3010 \\
\log(3) &= 0.4771
\end{align*}

\section{Differentiation}

\begin{equation}
\frac{d(c)}{dx} = 0 \quad \text{where } c \text{ is a constant}
\end{equation}

\begin{equation}
\frac{d(x^n)}{dx} = nx^{n-1}
\end{equation}

\begin{equation}
\frac{d(e^x)}{dx} = e^x
\end{equation}

\begin{equation}
\frac{d(a^x)}{dx} = a^x \ln(a) \quad \text{for example, } \frac{d(2^x)}{dx} = 2^x \ln(2)
\end{equation}

\begin{equation}
\frac{d(\ln(x))}{dx} = \frac{1}{x} \quad \text{remember, } \frac{d(\log(x))}{dx} = \frac{1}{x} \text{ only if base is } 'e'
\end{equation}

\begin{equation}
\frac{d(\sin(x))}{dx} = \cos(x)
\end{equation}

\begin{equation}
\frac{d(\cos(x))}{dx} = -\sin(x)
\end{equation}

\begin{equation}
\frac{d(\tan(x))}{dx} = \sec^2(x)
\end{equation}

\begin{equation}
\frac{d(\sec(x))}{dx} = \sec(x)\tan(x)
\end{equation}

\begin{equation}
\frac{d(\cot(x))}{dx} = -\csc^2(x)
\end{equation}

\begin{equation}
\frac{d(\csc(x))}{dx} = -\csc(x)\cot(x)
\end{equation}

\textbf{Product Rule:}

\begin{equation}
\frac{d[f_1 \cdot f_2 \cdot f_3]}{dx} = \frac{df_1}{dx} \cdot f_2 \cdot f_3 + \frac{df_2}{dx} \cdot f_1 \cdot f_3 + \frac{df_3}{dx} \cdot f_1 \cdot f_2
\end{equation}

\textbf{Quotient Rule:}

\begin{equation}
\frac{d\left[\frac{f_1}{f_2}\right]}{dx} = \frac{f_2 \cdot \frac{df_1}{dx} - f_1 \cdot \frac{df_2}{dx}}{(f_2)^2}
\end{equation}

\section{Limits}

\textbf{Indeterminate forms:} $0/0$, $\infty/\infty$, $\infty - \infty$, $\infty \times 0$, $\infty^0$, $0^0$, $1^\infty$

\textbf{Note:} "Infinity" is an indeterminate form, but it's not a defined number.

\textbf{Resolving limits using LHL and RHL:}

Step 1: Directly substitute the point of limit in $f(x)$. If the result is determinate, that is the value of the limit. If it's indeterminate, proceed with the next steps.

Step 2: Factorize and solve separately for the left-hand limit (LHL) and the right-hand limit (RHL).

\textbf{L'Hopital's rule:} If $f(a)/g(a)$ has the indeterminate form $0/0$ or $\infty/\infty$, then $\lim_{x \to a} \frac{f(x)}{g(x)} = \lim_{x \to a} \frac{f'(x)}{g'(x)}$.

\textbf{Note:} If the limit is in the form of $\infty - \infty$, we should combine the terms in some way and then apply L'Hopital's rule.

\textbf{Method of rationalization:} This method is often used to solve limits that involve square roots. By multiplying by the conjugate over itself, we can simplify the expression and find the limit.

\textbf{Trigonometric limits:}

\begin{equation}
\lim_{x \to 0} \frac{\sin(x)}{x} = 1
\end{equation}

\begin{equation}
\lim_{x \to 0} \frac{\cos(x) - 1}{x} = 0
\end{equation}

\begin{equation}
\lim_{x \to 0} \frac{\tan(x)}{x} = 1
\end{equation}

\begin{equation}
\lim_{x \to a} \frac{\sin[f(x)]}{f(x)} = 1 \quad \text{if } f(a) = 0
\end{equation}

\begin{equation}
\lim_{x \to a} \frac{\tan[f(x)]}{f(x)} = 1 \quad \text{if } f(a) = 0
\end{equation}

Examples:

\begin{equation}
\lim_{x \to 0} \frac{\sin(x^2)}{x} = \lim_{x \to 0} x \cdot \frac{\sin(x^2)}{x^2} = \lim_{x \to 0} x \cdot 1 = 0
\end{equation}

\begin{equation}
\lim_{x \to 0} \frac{\sin(x-1)}{x-1} = \sin(1)
\end{equation}
\textbf{Note:} sin(x-1)/(x-1) is not indeterminate, so we don't turn it to 1 and instead simply substitute the limit in.

\section{Integration}

\begin{equation}
\int dx = x + c
\end{equation}

\begin{equation}
\int ax^n dx = \frac{a x^{n+1}}{n+1} + c, \quad n \neq -1
\end{equation}

\begin{equation}
\int \frac{1}{x} dx = \ln|x| + c
\end{equation}

\begin{equation}
\int e^{mx} dx = \frac{e^{mx}}{m} + c
\end{equation}

\begin{equation}
\int a^{mx} dx = \frac{a^{mx}}{m \ln a} + c
\end{equation}

\begin{equation}
\int e^{\ln x} dx = x + c
\end{equation}

\begin{equation}
\int \sin x dx = -\cos x + c
\end{equation}

\begin{equation}
\int \cos x dx = \sin x + c
\end{equation}

\begin{equation}
\int \sec^2 x dx = \tan x + c
\end{equation}

\begin{equation}
\int \csc^2 x dx = -\cot x + c
\end{equation}

\begin{equation}
\int \sec x \tan x dx = \sec x + c
\end{equation}

\begin{equation}
\int \csc x \cot x dx = -\csc x + c
\end{equation}

\begin{equation}
\int \tan x dx = \ln |\sec x| + c
\end{equation}

\begin{equation}
\int \cot x dx = \ln |\sin x| + c
\end{equation}

\begin{equation}
\int \sec x dx = \ln |\sec x + \tan x| + c
\end{equation}

\begin{equation}
\int \csc x dx = -\ln |\csc x + \cot x| + c
\end{equation}

\begin{equation}
\int \frac{dx}{\sqrt{1-x^2}} = \sin^{-1} x + c
\end{equation}

\begin{equation}
\int \frac{dx}{1+x^2} = \tan^{-1} x + c
\end{equation}

\begin{equation}
\int \frac{dx}{\sqrt{1+x^2}} = -\cos^{-1} x + c
\end{equation}

\begin{equation}
\int e^x dx = e^x + c
\end{equation}

\section{Straight Lines}

\subsection{Section Formulas}

\textbf{Internal Section Formula:}

The coordinates of the point \(P\) which divides the line segment joining \(A(x_1, y_1)\) and \(B(x_2, y_2)\) internally in the ratio \(m:n\) are given by

\begin{equation}
P = \left(\frac{m x_2 + n x_1}{m + n}, \frac{m y_2 + n y_1}{m + n}\right)
\end{equation}

\textbf{External Section Formula:}

The coordinates of the point \(P\) which divides the line segment joining \(A(x_1, y_1)\) and \(B(x_2, y_2)\) externally in the ratio \(m:n\) are given by

\begin{equation}
P = \left(\frac{m x_2 - n x_1}{m - n}, \frac{m y_2 - n y_1}{m - n}\right)
\end{equation}

\subsection{Area Formulas}

\textbf{Area of a Triangle:}

The area of a triangle with vertices \(A(x_1, y_1)\), \(B(x_2, y_2)\), and \(C(x_3, y_3)\) is given by

\begin{equation}
\text{Area} = \frac{1}{2} [x_1(y_2 - y_3) + x_2 (y_3 - y_1) + x_3 (y_1 - y_2)]
\end{equation}

\textbf{Area of a Quadrilateral:}

The area of a quadrilateral with vertices \(A(x_1, y_1)\), \(B(x_2, y_2)\), \(C(x_3, y_3)\), and \(D(x_4, y_4)\) in order is given by

\begin{equation}
\text{Area} = \frac{1}{2} [x_1y_2 - x_2y_1 + x_2y_3 - x_3y_2 + x_3y_4 - x_4y_3 + x_4y_1 - x_1y_4]
\end{equation}

\subsection{Slope of a Line}

If 'm' is the slope of the line, then

\begin{equation}
m = \tan \theta = \frac{y_2 - y_1}{x_2 - x_1}
\end{equation}

Note: The slope of a line parallel to the x-axis is 0, and the slope of a line parallel to the y-axis is undefined (or infinity).

\subsection{Test for Collinearity of Three Points}

Three points are collinear if:

a) The area of the triangle formed by them is 0.

b) The slopes of the two lines they form are equal, i.e., the slope of AB equals the slope of BC.

\subsection{Equations of Line}

a) Line parallel to the x-axis:

\begin{equation}
y = b
\end{equation}

b) Line parallel to the y-axis:

\begin{equation}
x = a
\end{equation}

c) Line passing through \((x_1, y_1)\) and having slope 'm':

\begin{equation}
y - y_1 = m(x - x_1)
\end{equation}

d) Slope and y-intercept form:

\begin{equation}
y = mx + c
\end{equation}

e) Two point form:

\begin{equation}
y - y_1 = \frac{y_2 - y_1}{x_2 - x_1} (x - x_1)
\end{equation}

f) Intercept form:

\begin{equation}
\frac{x}{a} + \frac{y}{b} = 1
\end{equation}

g) Normal form, where the line has a perpendicular distance from the origin of 'P', and that perpendicular makes an angle \(\theta\) with the x-axis:

\begin{equation}
x \cos \theta + y \sin \theta = P
\end{equation}

\subsection{Angle Between Lines}

The angle between two lines is given by

\begin{equation}
\tan \theta = \left| \frac{m_1 - m_2}{1 + m_1 m_2} \right|
\end{equation}

Note: If the lines are parallel, \(m_1 = m_2\). If the lines are perpendicular, \(m_1 m_2 = -1\).

Tip: The equation of any line parallel to \(ax + by + c = 0\) can be written as \(ax + by + k = 0\).

\subsection{Condition for a Line to be Equally Inclined from Two Given Lines}

A line is equally inclined from two given lines if

\begin{equation}
\frac{m - m_1}{1 + m m_1} = -\frac{m - m_2}{1 + m m_2}
\end{equation}

\subsection{Equation of Angle Bisector of Two Lines}

The equation of the angle bisector of two lines is given by

\begin{equation}
\frac{a_1 x + b_1 y + c_1}{\sqrt{a_1^2 + b_1^2}} = \pm \frac{a_2 x + b_2 y + c_2}{\sqrt{a_2^2 + b_2^2}}
\end{equation}

Note: Angle bisectors of two lines are perpendicular. If \(a_1 a_2 + b_1 b_2 > 0\), then the '+' sign gives the obtuse angle bisector, and the '-' sign gives the acute angle bisector. If \(a_1 a_2 + b_1 b_2 < 0\), then vice versa.

\subsection{Perpendicular Distance from a Point to a Line}

The perpendicular distance 'd' from a point \((x_1, y_1)\) to the line \(ax + by + c = 0\) is given by

\begin{equation}
d = \frac{|ax_1 + by_1 + c|}{\sqrt{a^2 + b^2}}
\end{equation}

\subsection{Foot of Perpendicular from a Point to a Line}

The coordinates of the foot 'F' of the perpendicular drawn from a point 'P' \((x_1, y_1)\) to the line \(ax + by + c = 0\) can be found using:

Method 1: Calculate the equation of PF and find the point of intersection.

Method 2: Let \(ax + by + c= 0\) be \(a\alpha + by + c\) (\(\alpha\) is the x-coordinate), then the y-coordinate is \(-\frac{a\alpha + c}{b}\). So, \(F = \left(\alpha, -\frac{a\alpha + c}{b}\right)\).

Method 3:

\begin{equation}
\frac{x - x_1}{a} = \frac{y - y_1}{b} = -\frac{ax_1 + by_1 + c}{a^2 + b^2}
\end{equation}

where \(F(x, y)\) is the foot of the perpendicular.

\section{Circles}

\subsection{Standard equation of a circle}
For a circle with center \((\alpha, \beta)\), the equation is:
\[
(x-\alpha)^2 + (y-\beta)^2 = r^2
\]
where \((x,y)\) is any point on the circle.

\subsection{General equation of a circle}
The general equation of a circle is:
\[
x^2+y^2+2gx+2fy+c = 0
\]
Note: the coefficient of \(x^2\) must be equal to the coefficient of \(y^2\), else there is no circle.
The center is \((-g, -f)\) and the radius is \(\sqrt{g^2+f^2-c}\).

\subsection{Family of circles}

\subsubsection{Circle passing through two points}
A circle passing through \(A(x_1, y_1)\) and \(B(x_2, y_2)\) can be represented as:
\[
(x-x_1)(x-x_2) + (y-y_1)(y-y_2) + \lambda L = 0
\]
where \(L\) is the equation of the line joining \(A\) and \(B\), and \(\lambda\) is a parameter.

\subsubsection{Circle touching a line}
A family of circles touching the line \(ax+by+c = 0\) can be represented as:
\[
(x-x_1)^2 + (y-y_1)^2 + \lambda(ax+by+c) = 0
\]

\subsubsection{Circle passing through the interior of another circle and a line}
A family of circles passing through the interior of circle \(S = x^2+y^2+2gx+2fy+c\) and line \(L = 0\) can be represented as:
\[
S + \lambda L = 0
\]
or
\[
x^2+y^2+2gx+2fy+c+\lambda(ax+by+c)=0
\]

\subsubsection{Circle through intersection of two circles}
A family of circles passing through the intersection of circles \(S_1=x^2+y^2+2g_1x+2f_1y+c_1=0\) and \(S_2=x^2+y^2+2g_2x+2f_2y+c_2=0\) can be represented as:
\[
x^2+y^2+2g_1x+2f_1y+c_1+\lambda(x^2+y^2+2g_2x+2f_2y+c_2)=0
\]

\subsection{Relative position of a point and a circle}
The relative position of a circle \(S = x^2+y^2+2gx+2fy+c\) and a point \((x_1, y_1)\) can be determined as:

If 
\[
S = x_1^2+y_1^2+2gx_1+2fy_1+c = 0
\]
then the point is on the circle.

If 
\[
S = x_1^2+y_1^2+2gx_1+2fy_1+c > 0
\]
then the point is outside the circle.

If 
\[
S = x_1^2+y_1^2+2gx_1+2fy_1+c < 0
\]
then the point is inside the circle.


\end{document}